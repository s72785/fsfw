\documentclass{beamer}              % Für Präsentation
% \documentclass[handout]{beamer}   % Für Handouts
%   \mode<handout>{%
%   \usepackage{pgfpages}
%   \pgfpagesuselayout{2 on 1}[a4paper,border shrink=5mm]
% }
 
\usepackage[T1]{fontenc}
\usepackage[utf8]{inputenc}
\usepackage[ngerman]{babel}
\usepackage{hyperref}
\usepackage{longtable}
\usepackage{graphicx}
\usepackage{makeidx}

\title{Vortrag Freie Software}
\subtitle{Mehr Freiheit für Forschung und Lehre}
%\author{W}
%\institute{StuRa der HTW Dresden, Bereich Datenkultur}
\titlegraphic{\includegraphics[keepaspectratio=true, scale=1]{wikischool-logo}}
\subject{Freie Software für Freie Studenten}
\keywords{Freie Software, Open Source, Schulung, StuRa, Einarbeitung, Studium, Forschung, Lehre, Zusammenarbeit, Kollaboration}
\date{\today}

\makeindex

\begin{document}

% \frame{\titlepage}
% \frame[plain]{\titlepage}
% \frame[plain]{\maketitle}

\begin{frame}
%   \maketitle
  \begin{center}
    %\includegraphics[keepaspectratio=true, scale=1]{fsf-gnu-logo}
    \begin{longtable}{|l|l|}
      \hline
      
       \textbf{Thema:}        &  \textbf{Gemeinsame Wissensverwaltung per Wiki}                  \\ 
      \hline       
       Ort:                   &  KoSe, Hohenstein                                                \\ 
      \hline       
       Datum:                 &  10.01.2015                                                      \\
      \hline       
       Zeit:                  &  ??? Uhr                                               \\ 
%      \hline       
%       Referent:              &  \href{http://www.stura.htw-dresden.de/members/}{W}  \\ 
      \hline       
    \end{longtable}
  \end{center}
\end{frame}




\begin{frame}
  \frametitle{Inhalt}
  \tableofcontents
\end{frame}

\AtBeginSection{%
  \tableofcontents[currentsection]
}

\section{Woher kommt das?}

%\begin{frame}
%  \frametitle{Was sind Wikis}
%\end{frame}

%\begin{frame}
%  \frametitle{Beispiele}
%\end{frame}



%\section{Die Vier Freiheiten}

%\section{In }

%Der Falsche Ansatz: Open Source

%Offene Formate


\begin{frame}
  \frametitle{Danke für eure Aufmerksamkeit!}
  \framesubtitle{Gibts Fragen?}

  \begin{figure}
%    \begin{center}
      \includegraphics[scale=0.5]{Circle-question_400x400.png}
      \label{fig:Fragezeichenschild}
%    \end{center}
  \end{figure}

\end{frame}


\begin{frame}
  \frametitle{Quellen}
  \begin{itemize}
    \item \url{http://wiki.stura.htw-dresden.de}
    \item \url{https://www.c3d2.de}
    \item \url{http://fsfe.org/}
    \item \url{http://eff.org/}
    \item \url{http://media.ccc.de/browse/congress/2014/31c3_-_6123_-_en_-_saal_1_-_201412291130_-_freedom_in_your_computer_and_in_the_net_-_richard_stallman.html#video}{RMS auf dem 31c3}
    
  \end{itemize}  
\end{frame}

% \begin{frame}         %% Versuch eines Indexes
%   \frametitle{Index}
% 
%   \index{Haupteintrag!Untereintrag}           %% Besp.einträge (sollten eigentlich irgendwo im Text stehen)
%   \index{Haupteintrag!weitererUntereintrag}
%   \index{weitererHaupteintrag!noch_ein_Untereintrag}
%   \printindex
% \end{frame}
% 
% \begin{frame}
%   \frametitle{Glossary}
%
%   Hello world
%
%   \makeglossary
% \end{frame}


%\begin{frame}
%  \frametitle{Das Wiki ist, was ihr draus macht!}
%  \framesubtitle{Link zur Präsentation}
%  \begin{itemize}
%    \item \url{}
%  \end{itemize}
%\end{frame}

\end{document}
